\documentclass{acm_proc_article-sp}
\usepackage[utf8]{inputenc}
\usepackage[brazil]{babel}
\usepackage{hyperref}
\usepackage{color}

\newcommand{\remove}[1]{}

\hyphenation{tra-zen-do Bra-sil}

\numberwithin{equation}{section}

\begin{document}

\title{Sem Título}

\numberofauthors{1}
\author{
\alignauthor
Iam Jabour  
\and \alignauthor \email{ijabour@inf.puc-rio.br}
}


\maketitle

\begin{abstract}


\end{abstract}

\section*{RESUMO}\normalsize %\the\parskip \the\baselineskip%\ninept
%\begin{abstract}


%\end{abstract}


% A category with the (minimum) three required fields
%\category{H.4}{Information Systems Applications}{Miscellaneous}
%A category including the fourth, optional field follows...
%\category{D.2.8}{Software Engineering}{Metrics}[complexity measures, performance measures]

%\terms{Delphi theory}

\keywords{Aprendizado de Máquina, Extração de Informação, Heurística}

\section{Introdução}

A {\it World Wide Web} estendeu o paradigma de pesquisa que era conhecido, pois
as técnicas necessárias para buscar em seu acervo, cada dia maior, tiveram 
de ser aperfeiçoadas.
	A tarefa de Extração de Informação ({\it Informarion Retrieval}, IR)
apresenta estratégias que ajudam nos procedimentos de pequisa, bunscando
tornar a informação disponível mais estruturada o que facilita seu entendimento.

Os documentos da Web trazem informação de forma semi-estuturada e,
	mutas vezes, junto a um conjunto excessivo de conteúdo não relevante, 
	como propaganda, menu ou links relacionados. Tornar esses documentos 
	menos poluidos, retirando uma parcela de informação irelevante desses, é 
	uma tarefa de valor para as ferramentas de busca.
	Com isso, 
	alguns dos estudos da área de {\it IR} em documentos da Web são 
	direcionados a tarefa de remover o {\bf conteúdo não relevante} ou 
	encontrar o {\bf conteúdo relevante} de um documento, no esforço de 
	filtrar a informação para diminuir a quantidade de dados não
	informativos processados.

{\bf Conteúdo relevante} é a informação trazida únicamente pela página,
		ou seja, 
		o conteúdo que motivou a criação do documento. 
	Complementarmente,
	{\bf conteúdo não relevante} é a informação apresentada para facilitar a 
	navegação, ajudar na aparência ou trazer informações que não se 
	referem ao conteúdo relevante,
		ou seja, 
		todo o tipo de informação que normalmente aprerece repetidamente 
		dentre vários documentos de um Site.

Muitos dos estudos direcionados a encontrar o conteúdo não relevante,
apresentam métodos para detectar o {\it tamplate} do documento.
	Essa tentativa devesse a proporção de informação referente ao 
	{\it tamplates} nos documentos da Web, confirmada por \cite{Gibson2005}
	que apresenta estudos onde cerca de $40\%$ a $50\%$ da
	informação dos documentos é constituida por {\it tamplate}.

Outra abordagem, também muito utilizada, é a criação de regras/heurísticas para
detectar o conteúdo relevante.
	Nessas abordagens
		o domínio da aplicação é bem direcionado, pois o tipo de informação é 
		bem definido.
		Como exemplo,
			existem métodos expecializados em detectar o corpo de uma notícia 
			dentro de uma página de jornal, ou site semelhante,
			anotando qual parte da notícia é 
				o título,
				o autor, o corpo/texto e
				a data de publicação.
			Existem referências para estudos com domínios como:
				notícia {\bf \cite{}...}, como o objetivo exemplificado acima;
				comércio eletrônico {\bf \cite{}...}, com o objetivo de 
				encontrar os produtos a descrição e o preço oferecido;
				blog {\bf \cite{}...}, com o objetivo de separar cada 
				{\it post} e seus comentários.


\remove{
Embora estudos como \cite{}... apresentem resutados "interessantes" na detecção de templates, 
%\subsection{Comparação com os resultados}
}



% \balancecolumns

\bibliographystyle{alpha}
\bibliography{bib}
\end{document}
