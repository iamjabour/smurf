\documentclass{acm_proc_article-sp}
\usepackage[utf8]{inputenc}
\usepackage[brazil]{babel}
\usepackage{hyperref}
\usepackage{color}

\newcommand{\remove}[1]{}

\hyphenation{tra-zen-do Bra-sil}

\numberwithin{equation}{section}

\begin{document}

\title{Sem Título}

\numberofauthors{1}
\author{
\alignauthor
Iam Jabour  
\and \alignauthor \email{ijabour@inf.puc-rio.br}
}


\maketitle

\begin{abstract}


\end{abstract}

\section*{RESUMO}\normalsize %\the\parskip \the\baselineskip%\ninept
%\begin{abstract}


%\end{abstract}


% A category with the (minimum) three required fields
%\category{H.4}{Information Systems Applications}{Miscellaneous}
%A category including the fourth, optional field follows...
%\category{D.2.8}{Software Engineering}{Metrics}[complexity measures, performance measures]

%\terms{Delphi theory}

\keywords{Aprendizado de Máquina, Extração de Informação, Heurística}

\section{Introdução}

A {\it World Wide Web} estendeu o paradigma de pesquisa que era conhecido, 
	introduzindo o conceito de busca a milhares de pessoas.
A tarefa de Extração de Informação ({\it Informarion Retrieval}, IR) apresenta 
	estratégias que ajudam nos procedimentos de pequisa, estruturando a 
	informação disponível na tentativa de facilitar seu entendimento.

Os documentos da Web por apresentarem a informação de forma semi-estuturada,
	mutas vezes, junto a um conjunto excessivo de conteúdo não informativo
	representam um grande desafio para a extração de informação. 
Por este motivo,
	tornar os documentos menos poluidos junto a uma estutura básica
	é uma tarefa de valor com diversos estudo na útima década.
	\remove{Retirar uma parcela de informação não relevante desses documentos 
	proporciona benefícios claros.}

Tradicionalmente, existem duas formas básicas de filtar a informação nos 
	documento da Web, diminuindo a quantidade de dados não informativos 
	processados.
A primeira é direcionada a encontrar o {\bf conteúdo relevante} de um 
	documento. {\bf Conteúdo relevante} é a informação trazida únicamente pelo 
	documento, ou seja,
		o conteúdo que motivou a criação do documento.

Complementarmente, a segunda abordagem é direcionada a encontrar e remover o 
	{\bf conteúdo não relevante} de um documento. {\bf Conteúdo não relevante} 
	é a informação que existe somenta para facilitar a navegação,
	ajudar na aparência ou trazer informações que não sejam referentes ao 
	conteúdo relevante, ou seja,
		todo o tipo de informação que normalmente aprerece repetidamente 
		dentre vários documentos de um Site.


Estudos como [...] direcionados a encontrar o conteúdo não relevante,
apresentando métodos para detectar o {\it tamplate} do documento.
	Essa tentativa devesse a proporção de informação referente ao 
	{\it tamplates} nos documentos da Web, citada por estudos como o de 
	[Brooks2003] e confirmada por análises como a de \cite{Gibson2005}
	que apresenta estudos onde cerca de $40\%$ a $50\%$ da
	informação dos documentos é constituida por {\it tamplate}.


Uma abordagem comum para detectar o conteúdo relevante é a criação de
regras/heurísticas. 
	Nessas 
		o domínio da aplicação é bem direcionado, pois o tipo de informação é 
		bem definido.
		Como exemplo,
			existem métodos expecializados em detectar o corpo de uma notícia 
			dentro de uma página de jornal, ou site semelhante,
			anotando qual parte da notícia é 
				o título,
				o autor, o corpo/texto e
				a data de publicação.
			Existem referências para estudos com domínios como:
				notícia {\bf \cite{}...}, como o objetivo exemplificado acima;
				comércio eletrônico {\bf \cite{}...}, com o objetivo de 
				encontrar os produtos, a descrição e o preço oferecido;
				blog {\bf \cite{}...}, com o objetivo de separar cada 
				{\it post} e seus comentários.


\remove{
Embora estudos como \cite{}... apresentem resutados "interessantes" na detecção de templates, 
%\subsection{Comparação com os resultados}
}



% \balancecolumns

\bibliographystyle{alpha}
\bibliography{bib}
\end{document}
